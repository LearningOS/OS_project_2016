\documentclass[11pt]{beamer}
\usepackage{amsmath,amssymb}
\usepackage{latexsym}
\usepackage{fancyhdr}
\usepackage{verbatim}
\usepackage{color}
\usepackage{array}
\usepackage{multirow}
\usepackage{graphicx}
\usepackage{hyperref}
\usepackage{xeCJK}
\usepackage{beamerthemeshadow}	%阴影样式
\usepackage{xunicode}
\usepackage{bm}
\usepackage{cite}
\usepackage{multirow}
\usepackage{geometry}
\usepackage{longtable}
\usepackage{algorithm}
\usepackage{algorithmic}
\usepackage{booktabs}
\setCJKmainfont{Hiragino Sans GB W3}

\setbeamertemplate{navigation symbols}{}	%修正工具图样
\useoutertheme{infolines}	%显示作者
\usetheme{Berkeley}

\title{操作系统 \quad 第二次报告}
\author{李宇轩}
\date{\today}
\date{}

\begin{document}
\maketitle

\section{gdb}

\begin{frame}
\frametitle{实验目标描述}
\begin{itemize}
\item 完成一个硬件模块的debugger,可以和gdb通讯
\item 硬件相关的杂事
\end{itemize}
\end{frame}

\begin{frame}
\frametitle{已有相关工作介绍}
\begin{itemize}
\item 张宇翔组的naive\_mips
\end{itemize}
\end{frame}

\begin{frame}
\frametitle{小组成员分工}
\begin{itemize}
\item 硬件(串口,flash),gdb debug
\end{itemize}
\end{frame}

\begin{frame}
\frametitle{实现方案·串口}
\begin{itemize}
\item 利用边沿同步和脉冲同步做异步时钟域同步
\item 可以运行在460800的速率下,和回环测试达到的效率一致
\item 通过连续收发3个小时的测试
\end{itemize}
\end{frame}

\begin{frame}
\frametitle{实现方案·Flash}
\begin{itemize}
\item 实现了基本的nor\_flash操作,并为了单元测试用HDL封装一些基本操作
\item 通过了读写测试,稳定性测试,效率和datasheet上描述的一致
\end{itemize}
\end{frame}

\begin{frame}
\frametitle{实现方案·gdb debugger}
\begin{itemize}
\item 调研,进一步,工业界标准-JTAG、Gdb Server
\item 实验,qemu、naive\_mips server手动交互
\item 基础,流水线暂停逻辑,控制MMU
\item 扩展,debug中断,可以执行指令
\end{itemize}
\end{frame}

\begin{frame}
\frametitle{演示}
\end{frame}

\begin{frame}
\frametitle{后续}
\begin{itemize}
\item 如有必要对Flash模块的接口进行一定的修改
\item 支持软中断,支持从串口读取指令执行
\item 修复源码级调试的存在的问题
\item 解决虚实地址问题
\item 向mips的EJTAG标准靠拢
\end{itemize}
\end{frame}

\end{document}
